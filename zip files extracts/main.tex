\documentclass{article}
\usepackage[utf8]{inputenc}
\usepackage{plcourse}
\usepackage{ttquot}
\usepackage{proof}
\usepackage{stmaryrd}
\usepackage{angle}
\usepackage{amssymb}
\usepackage{hyperref}
\usepackage[bluebook,boxanswers]{answer}
\usepackage{graphicx}
\usepackage{tikz}
\usepackage{tikz-qtree}
\usepackage{xspace}
\usepackage{xcolor}
\usepackage{listings}

%% Synatx highlighting for OCaml
\lstset{
 language=caml,
 columns=[c]fixed,
% basicstyle=\small\ttfamily,
 keywordstyle=\bfseries,
 upquote=true,
 commentstyle=,
 breaklines=true,
 showstringspaces=false,
 stringstyle=\color{blue},
 literate={'"'}{\textquotesingle "\textquotesingle}3
}


\homework{4}

%\newcommand{\largestep}[1]{#1\textsubscript{}} 
\newcommand{\largestep}[1]{\rulename{#1${}_{\text{Lrg}}$}}
\newcommand{\tlet}{\textbf{let }}
\newcommand{\tin}{\textbf{in }}
\newcommand{\tref}{\textbf{ref }}
\newcommand{\tint}{\textbf{int }}

\begin{document}
\hwsubheader

\vspace{5cm}
{\LARGE
\begin{tabular}{rp{0.6\linewidth}}
  Name:&\todo{Add your name here}\\
  UML ID:&\todo{Add your student ID here}\\
  Collaborators:&{\normalsize
    \todo{Put your collaborators here, if any.}
    }
\end{tabular}
}

\vfill
\textit{Make sure that the remaining pages of this assignment do not contain any identifying information.}
\vfill
%%% NOTE:
% %%  Use this syntax to write inference rules
      % %%   \infrule[Name]
      % %%      {
      % %%       a_1 \in A \quad a_2 \in A
      % %%      }
      % %%      {
      % %%       a \in A
      % %%      }
      % %%      {
      % %%       side condition
      % %%      }
     % %%

     
     %% You can replace <RULE-NAME> with the name of
     %% actual rule.
     
     %% Use \dots if you need to add something like 
     %% "..." to the end of your small step semantics

     %% Follow this pattern to write lareg step semantics
     %% Replace the condition as necessary
     %%\begin{center}
     %%\infrule[\largestep{RULE-NAME}]{a \stepsto v}
     %%{\cond{\TRUE}{a}{\FALSE} \stepsto v }{side-condition}
     %%\end{center}


     %%% Use this for structuring proofs.
     % \begin{description}
     %     \item[case 1]: ...
     %     \item[case 2]: ....
     %     \item[case n]: ...
     % \end{description}

     %% Use this to write and state theorems.
     % \begin{theorem}
     %    [Insert theorem statement]
     % \end{theorem}

    %% Use this to wrap your proofs.
    % \begin{proof}
    %     [Insert proof]
    % \end{proof}

    % For more help please refer to this link:
    % https://www.overleaf.com/learn/latex/Lists



\newpage


\begin{question}{Equivalence of Semantics}{(40 points)}
  Recall the grammar for the calculator language used from HW2 and HW3. In this question, you will prove the equivalence of large-step and multi-step semantics.

  \medskip
  % \textbf{NOTE:} You have to use/fix your small-step and large-step inference rules from your HW2 and HW3 and use them in the proof. You \textbf{must} copy/write all the inference rules again for reference. During grading, we will not refer to your previous ``HW2'' or ``HW3'' submissions. As such, your current submission must be self-contained.
  
    \[
    \begin{array}{l l l}
      n & \in & \Z \\
      a &::= & n \mid a_1 + a_2 \mid a_1 \times a_2 \\
      b &::= & \TRUE \mid \FALSE \mid a = a \mid a \ne a \\
       & & \mid a \le a \mid a \gt a \mid \neg b \mid b \&\& b \\
      v & ::= & n \mid \TRUE \mid \FALSE
    \end{array}
    \]


  \begin{subquestion} Prove the following theorem using induction.
    \begin{theorem*}[Equivalence of Semantics for Arithmetic Expressions] For all expressions $a$,
integers $n$, we have:
\[ a \stepsto n \iff a \stepsone^* n   .\]
\end{theorem*}

    You need to prove the theorem in both the directions. That is, you need to provide proofs for the following statements.
    \[\text{If } a \stepsto n \text{ then }  a \stepsone^* n   \]
    and
    \[\text{If } a \stepsone^* n  \text{ then }  a \stepsto n   .\]

    Colloquially, we refer to the former statement as forward direction and the latter as the reverse direction.
    
      \begin{subsubquestion}
      What is the set that you will induct on in the forward direction?
      \end{subsubquestion}

    \begin{subsubquestion}
      What is the property that you will induct on in the forward direction? Formally state the property.

      \medskip
      \textit{\textbf{Hint:}} See Lecture 04 for an example formal statement.
    \end{subsubquestion}

    \begin{subsubquestion}
      Write the inductive reasoning principle for the forward direction.
    \end{subsubquestion}

    \begin{subsubquestion}
      Provide the actual proof for both forward and reverse directions.
      \begin{proof}
        %Proof should go here.
      \end{proof}
    \end{subsubquestion}
    
  \end{subquestion}

  \begin{subquestion} Prove the following theorem using induction.
\begin{theorem*}[Equivalence of Semantics for Boolean Expressions] For all expressions $a$,
values $v \in \{\TRUE, \FALSE \}$, we have:
\[ b \stepsto v \iff b \stepsone^* v   .\]
\end{theorem*}
      \begin{subsubquestion}
      What is the set that you will induct on in the forward direction?
      \end{subsubquestion}

    \begin{subsubquestion}
      What is the property that you will induct on in the forward direction? Formally state the property.
      
    \end{subsubquestion}

    \begin{subsubquestion}
      Write the inductive reasoning principle for the forward direction.
    \end{subsubquestion}

    \begin{subsubquestion}
      Provide the  proof for both forward and reverse directions. Note that you have to handle the proof for all cases.
      You cannot state ``same as above'' or ``similar to the case''.

      \medskip
      \textit{\textbf{Hint:}} When you handle cases that involve comparison between arithmetic expressions, you may need to invoke the lemmas proved in the previous part of the question.
    \begin{proof}
       %Proof should go here.
    \end{proof}
    \end{subsubquestion}

\end{subquestion}
  
\begin{answer}{}
\end{answer}

\end{question}
\end{document}
