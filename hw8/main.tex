%Given to students
\documentclass{article}
\usepackage[utf8]{inputenc}
\usepackage{plcourse}
\usepackage{answer}
\usepackage{ttquot}
\usepackage{proof}
\usepackage{stmaryrd}
\usepackage{angle}
\usepackage{amssymb}
\usepackage{amsmath}
\usepackage{hyperref}
\usepackage{graphicx}
\usepackage{tikz}
\usepackage{tikz-qtree}
\usepackage{xspace}
\usepackage{xcolor}
\usepackage{listings}


\homework{8}


\begin{document}
\hwsubheader

\vspace{5cm}
{\LARGE
\begin{tabular}{rp{0.6\linewidth}}
  Name:&\todo{Add your name here}\\
  UML ID:&\todo{Add your student ID here}\\
  Collaborators:&{\normalsize
    \todo{Put your collaborators here, if any.}
    }
\end{tabular}
}

\vfill
\textit{Make sure that the remaining pages of this assignment do not contain any identifying information.}
\vfill

\newpage



% Use the \infrule macro for neatly typeset inference rules.
% Syntax:
% \infrule[<RULE-NAME>]  % Replace <RULE-NAME> with the rule's name
%    {<Premises>}         Premises of the rule go here
%    {<Conclusion>}       Conclusion of the rule
%    {<Side Condition>}   Side condition, if any (leave blank if not needed)

% Use macros for writing typing judgments directly:
% Uncomment to see how this render
% $\Gamma \vdash e : \tau ~ \triangleright ~ $C

% Try to see how $\Gamma, \vdash, \tau$ render


\begin{question}{Type Inference}{(30 points)}

% Uncomment to see how this rule renders
% \infrule[A]
% {\Gamma, x \mapsto \tau, f \mapsto X \vdash e_1 : \tau_1 \quad \Gamma, f \mapsto X \vdash e_2 : \tau_2}
% {\Gamma \vdash \texttt{A } f = \lambda x:\tau. e_1 \texttt{ in } e_2 : \tau_2}
% {where X = $\tau \mapsto \tau_1$}

\begin{subquestion} {Inference rules for pairs and projection}
    % Start your answer here
\end{subquestion}

\begin{subquestion} {Inference rules for conditionals and integer inequality}
    % Start your answer here
\end{subquestion}

\begin{subquestion} {Inference rules for let}
    % start your answer here
\end{subquestion}

\end{question}

\end{document}
