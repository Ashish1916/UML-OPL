%% Given to students
\documentclass{article}
\usepackage[utf8]{inputenc}
\usepackage{plcourse}
\usepackage{answer}
\usepackage{ttquot}
\usepackage{proof}
\usepackage{stmaryrd}
\usepackage{angle}
\usepackage{amssymb}
\usepackage{hyperref}
\usepackage{graphicx}
\usepackage{tikz}
\usepackage{tikz-qtree}
\usepackage{xspace}
\usepackage{xcolor}
\usepackage{listings}


\homework{5}


\begin{document}
\hwsubheader

\vspace{5cm}
{\LARGE
\begin{tabular}{rp{0.6\linewidth}}
  Name:&\todo{Add your name here}\\
  UML ID:&\todo{Add your student ID here}\\
  Collaborators:&{\normalsize
    \todo{Put your collaborators here, if any.}
    }
\end{tabular}
}

\vfill
\textit{Make sure that the remaining pages of this assignment do not contain any identifying information.}
\vfill

\newpage


\begin{question}{Substitution}{(20 points)}
%%  Uncomment the following lines to see how they render in PDF.
  
%% Inductive definition of capture-avoiding substitution.
%% \begin{align*}
%%     \subst{y}{e}{x} & =
%%     \begin{cases}
%%       e & \text{if $x = y$}\\
%%       y & \text{if $x \ne y$}
%%    \end{cases} \\
%% \subst{(e_1~e_2)}{e}{x} &= \subst{e_1}{e}{x}~\subst{e_2}{e}{x}\\
%% \subst{(\lam{y}{e'})}{e}{x} &=
%% \begin{cases}
%% \lam{y}{e'} & \text{if $x = y$}\\
%% \lam{y}{(\subst{e'}{e}{x})} & \text{if $x \ne y$ and $y\not\in\mathit{FV}(e)$}\\
%% \lam{z}{(\subst{(\subst{e'}{z}{y})}{e}{x})} &
%% \text{if $x \ne y$ and $y\in\mathit{FV}(e)$, where} \\
%%  & \text{ \qquad $z \not\in \mathit{FV}(e) \cup \mathit{FV}(e') \cup \{x\}$} \\
%% \end{cases}
%% \end{align*}

% To write free variables use $\mathit{FV}(e)$ to represent free variables in an expression e.
% To write union operator use \cup.
% To write lambda terms, use: \lam{x}{body}
% Example: \lam{x}{x~y} produces λx.x y
% To write a lambda use \lam{y}{a~b} where {a~b} is the body of this lambda which is an application.
% Use \ne to write ≠ , \in to write ∈ , and \not \in to write ∉.

  \begin{subquestion}
    \todo{  Write your solution here.}
    

% Use \mathit{FV} while writing the inductive definition using the following format:
% \begin{align*}
%     \mathit{FV}{case 1}
%     \mathit{FV}{case 2}
%     \mathit{FV}{case 3}
% \end{align*}
  
\end{subquestion}


\begin{subquestion}

  \todo{Write your solution here.}
  
 %%  \begin{enumerate}
 %%  \item $\subst{(\lam{z}{y~\lam{y}{y~z~w}})}{(\lam{x}{x})}{y}$
 %%  \item $\subst{((\lam{x}{x~y})~(\lam{z}{x~z}))}{(\lam{w}{w~w})}{x}$
 %%  \item $\subst{(\lam{y}{x~y})}{(\lam{z}{y~z})}{x}$
 %%  \item $\subst{((\lam{x}{\lam{w}{w~x}})~\lam{y}{x~y})}{(w~w)}{x}$
 %% \end{enumerate}
 

 % Show all steps and then final result for each.
 % You can use the enumerate or itemize environment to have it display as a list. Enumerate is given to you.
 % Use: $\subst{(\lam{a}{a})} {a}{b}$ where b is the variable or expression that you're replacing for a in the expression \lambda a.a
 % Start writing answer here
 \begin{enumerate}
     \item 
     \item 
     \item
     \item 
 \end{enumerate}
\end{subquestion}

\begin{subquestion}
% Note: You need to find substitution examples that produce different results under the original vs alternate definitions.
  \todo{Write your solution here.}


\end{subquestion}

\end{question}

\end{document}
